% Modelo para relatório da disciplina de Projecto de Engenharia Informatica
% (adaptado para qualquer mestrado)

% Incorpora elementos impostos pelo Regulamento de Estudos Pos-Graduados da
% Universidade de Lisboa (Deliberacao 1506/2006 - Diario da República, 2.a série 
% - n.o 209 - 30 de Outubro de 2006)

% ----------------------------------------------------------------------------------------------
% ---------------------- CLASSE E TAMANHO DE LETRA DO DOCUMENTO --------------------------------
% ----------------------------------------------------------------------------------------------

\documentclass[11pt,openright,twoside]{report}

% ----------------------------------------------------------------------------------------------
% --------------------------------- PACKAGES ESSENCIAIS ----------------------------------------
% ----------------------------------------------------------------------------------------------

% Conversao Caracteres
\usepackage[utf8]{inputenc}
% Quem tiver problemas com os acentos, trocar utf8 por latin1

% Linguagen(s)
\usepackage[portuguese,english]{babel}

% Tipo de letra (ex: Times New Roman)
\usepackage{times}

% Inserir Figuras
\usepackage{graphicx}

% Indice remissivo
\usepackage{makeidx}
\makeindex % usar este comando para inserir indice

% Glossario
\usepackage{glossaries}
\makeglossary % usar este comando para inserir glossario

% Links
\usepackage{hyperref}

% Package para cabecalhos
\usepackage{fancyhdr}
\usepackage{lastpage}

\fancyhf{} %
\lhead{\nouppercase {\leftmark}} %
\rhead{\nouppercase {\bf \thepage}}
\renewcommand{\headrulewidth}{0.1pt}

% Comando para inserir pagina em branco (inserida na numeracao, mas sem
% numero impresso) para quando e' preciso obrigar um capitulo a comecar
% do lado direito (pagina impar)
\newcommand{\LIMPA}{
\newpage
\mbox{}
\thispagestyle{empty}
}

% Igual, mas insere pagina com numero impresso (normalmente nao se usa)
\newcommand{\LIMPAC}{
\newpage
\mbox{}
\thispagestyle{plain}
}

% ----------------------------------------------------------------------------------------------
% --------------------------------- PACKAGES EXTRA/OPCIONAIS -----------------------------------
% ----------------------------------------------------------------------------------------------

% Margens de pagina
\usepackage[dvips]{geometry}
\geometry{a4paper=true,portrait=true,left=2.5cm,right=2.5cm,top=2.5cm,bottom=2.5cm}

% Espaçamento entre linhas
\usepackage{setspace}
\setstretch{1.15}
%\renewcommand{\baselinestretch}{1.15}

% Tamanho e espaçamento de figuras e tabelas
\usepackage{caption}
\captionsetup{font={small,stretch=1.0}}

% Contadores de figuras e tabelas
\usepackage{chngcntr}
\counterwithin{figure}{chapter}
\renewcommand{\thefigure}{\arabic{chapter}.\arabic{figure}}
\counterwithin{table}{chapter}
\renewcommand{\thetable}{\arabic{section}.\arabic{figure}}

% Contador de equaçoes
\usepackage{amsmath}
\numberwithin{equation}{chapter}

% Tamanho e espaçento dos footnotes
\renewcommand{\footnotesize}{\fontsize{9pt}{11pt}\selectfont}

% ----------------------------------------------------------------------------------------------
% --------------------------- ALTERAR INFORMAÇOES RELATIVAS AO PROJECTO ------------------------
% ----------------------------------------------------------------------------------------------

\newcommand{\TITULO}{TÍTULO DA DISSERTAÇÃO EM MAIÚSCULAS}
\newcommand{\Autor}{Nome completo do aluno}
\newcommand{\AutorNumAluno}{4xxxx}

%Orientador e CoOrientador *sem* titulos (e.g. Prof. Doutor)
\newcommand{\Orientador}{Nome Completo do Orientador}
\newcommand{\CoOrientador}{Nome Completo do Co-Orientador} %se nao se aplicar, nao importa o que aqui esteja

%Se aplicavel, o supervisor pode ter um titulo (Dr., Eng.) colocado aqui
\newcommand{\SupervisorInstituicao}{Nome Completo do Supervisor}  %se nao se aplicar, nao importa o que aqui esteja

\newcommand{\AnoLectivo}{2017/2018}
\newcommand{\Ano}{\Large{2018}}

% Comentar/descomentar conforme conveniente
%\newcommand{\TIPO}{DISSERTA\c{C}\~{A}O }
%\newcommand{\TIPO}{TRABALHO DE PROJETO }

% Comentar/descomentar conforme conveniente
\newcommand{\MESTRADO}{MESTRADO EM -- PREENCHER --}

% Comentar/descomentar conforme conveniente
\newcommand{\IdiomaTese}{\selectlanguage{portuguese}}
%\newcommand{\IdiomaTese}{\selectlanguage{english}}

% Comentar/descomentar conforme conveniente
%\newcommand{\Especializacao}{Especializa\c{c}\~{a}o em -- especializa\c{c}\~{a}o se aplic\'{a}vel --}
%\newcommand{\Especializacao}{Sistemas de Informa‹o}
%\newcommand{\Especializacao}{Intera‹o e Conhecimento}
%\newcommand{\Especializacao}{Engenharia de Software }

\usepackage{ifpdf}
\ifpdf
\pdfinfo {
	/Author (\Autor)
	/Title (\TITULO)
	/Subject (\MESTRADO)
	/Keywords () % inserir palavras chaves
	/CreationDate (D:\today)
}
\fi

% ----------------------------------------------------------------------------------------------
% --------------------------------------- INICIALIZAÇAO ----------------------------------------
% ----------------------------------------------------------------------------------------------

\begin{document}

\title{\TITULO}
\author{\Autor}
\date{\today}

\selectlanguage{portuguese}

\pagestyle{empty}

% ----------------------------------------------------------------------------------------------
% ---------------------------------------- CAPA ------------------------------------------------
% ----------------------------------------------------------------------------------------------

\begin{center}
\vspace{1cm}\normalfont\normalfont
\vfill
% Nome da Universidade
\textsc{\normalsize\uppercase{Universidade de Lisboa}}\\
% Nome da Faculdade
\normalsize\uppercase{Faculdade de Ciências}\\
% Nome do Departamento
\normalsize\uppercase{Departamento de -- PREENCHER --}\\
\vspace{1cm}
% Logotipo da Faculdade
\includegraphics[scale=.45]{images/logo_fcul.png}\\

\vspace{2.5cm}
\vfill
\IdiomaTese
\Large{\bf \TITULO}\\
\selectlanguage{portuguese}
\vspace{1.3cm}
%%
%% Eliminar na versao definitiva
\normalsize{\bf{Documento Provisório}}
%%
\vspace{1cm}
\vfill
\Large{\bf \Autor}\\
\vspace{1,8 cm}
\vfill
\large{\bf{\MESTRADO}}\\
%\normalsize{\Especializacao }\\
\vspace{2.3cm}
\vfill
\large{Orientação por:}\\
\large{Prof. \Orientador} \\
% DESCOMENTAR a linha relevante (se alguma), removendo o % no inicio
%e co-orientado pelo Prof. Doutor \CoOrientador \\
%e por \SupervisorInstituicao
\vspace{1.5 cm}
\vfill

%\vspace{1.5cm}
\vfill
\Ano
\end{center}

% ----------------------------------------------------------------------------------------------
% -------------------------------------- PAGINA DE ROSTO ---------------------------------------
% ----------------------------------------------------------------------------------------------

\newpage
\mbox{}
\newpage

% Inserir numeração romana
\setcounter{page}{1}
\pagenumbering{roman}

% ----------------------------------------------------------------------------------------------
% ----------------------------- DEDICATORIA/AGRADECIMENTOS -------------------------------------
% ----------------------------------------------------------------------------------------------

% Descomentar quando feito
%\input{chapters/dedicatoria}
%\newpage

% ----------------------------------------------------------------------------------------------
% -------------------------------- RESUMO PORTUGUES --------------------------------------------
% ----------------------------------------------------------------------------------------------

% Descomentar quando feito
%\input{chapters/resumo_pt}
%\newpage

% ----------------------------------------------------------------------------------------------
% -------------------------------- RESUMO INGLES -----------------------------------------------
% ----------------------------------------------------------------------------------------------

% Descomentar quando feito
%\input{chapters/resumo_eng}
%\newpage

% ----------------------------------------------------------------------------------------------
% ----------------------------------- INDICE ---------------------------------------------------
% ----------------------------------------------------------------------------------------------

\tableofcontents
\LIMPAC

% ----------------------------------------------------------------------------------------------
% ------------------------------- LISTA TABELAS ------------------------------------------------
% ----------------------------------------------------------------------------------------------

\listoftables
\LIMPAC

% ----------------------------------------------------------------------------------------------
% ------------------------------- LISTA FIGURAS ------------------------------------------------
% ----------------------------------------------------------------------------------------------

\listoffigures
\LIMPAC

% ----------------------------------------------------------------------------------------------
% ------------------------------- LISTA ABREVIATURAS -------------------------------------------
% ----------------------------------------------------------------------------------------------

% Descomentar quando feito
%\input{chapters/abreviaturas}
%\newpage

% ----------------------------------------------------------------------------------------------
% ------------------------------- TEXTO PRINCIPAL ----------------------------------------------
% ----------------------------------------------------------------------------------------------

\newpage
\setcounter{page}{0}
\pagenumbering{arabic}
\setcounter{page}{1}

% Texto template
\chapter{Lorem Ipsum}
\section{Lorem Ipsum}
\subsection{Lorem Ipsum}
Lorem ipsum dolor sit amet, consetetur sadipscing elitr, sed diam nonumy eirmod tempor invidunt ut labore et dolore magna aliquyam erat, sed diam voluptua. At vero eos et accusam et justo duo dolores et ea rebum. Stet clita kasd gubergren, no sea takimata sanctus est Lorem ipsum dolor sit amet. Lorem ipsum dolor sit amet, consetetur sadipscing elitr,  sed diam nonumy eirmod tempor invidunt ut labore et dolore magna aliquyam erat, sed diam voluptua. At vero eos et accusam et justo duo dolores et ea rebum. Stet clita kasd gubergren, no sea takimata sanctus est Lorem ipsum dolor sit amet. Lorem ipsum dolor sit amet, consetetur sadipscing elitr,  sed diam nonumy eirmod tempor invidunt ut labore et dolore magna aliquyam erat, sed diam voluptua. At vero eos et accusam et justo duo dolores et ea rebum. Stet clita kasd gubergren, no sea takimata sanctus est Lorem ipsum dolor sit amet.

% Descomentar quando feito
%\input{chapters/introducao}
%\input{chapters/chapter02}
%\input{chapters/chapter03}
% etc etc etc
%input{chapters/conclusao}

%\input{chapters/bibliografia}

%\input{chapters/anexo01}

\printglossary

\end{document}
